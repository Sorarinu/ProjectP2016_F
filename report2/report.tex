\documentclass[a4paper,10pt,fleqn]{jsarticle}
\usepackage{amsmath}
\usepackage[dvipdfm]{graphicx}


\title{プロジェクト実習テーマP CloudBM  中間レポート2}
\author{
  チームF \\
  C0114312 高畑 達也\\
  C0114015 新井 幸希\\
  C0114234 後藤 尚輝\\
  C0114088 上原 安里奈\\
}
\date{\today}


\begin{document}
\maketitle

\section{システム概要(新井)}

\section{プロトタイプの説明(高畑)}

\section{顧客インタビューとその解析(上原)}

\section{結論(後藤)}
プロトタイプを用いたインタビューの結果や
Googleフォームを用いた市場調査から一定数サービスを利用するユーザーがいることが分かった.
インタビューからは現状のプロトタイプに対する不満が明らかとなったためこれらに対処しサービスの質をこれから高めていく.

\subsection{今後の実装方針}
\subsubsection{フロント}
インタビュー結果から分かるように,現状のUIの完成度は低い.
そのため今後はある程度の機能拡張を図りつつも現状のプロトタイプからUIの完成度を高めていく. UIの完成度を高めるにあたっては自分たち以外にも感想を求めるといったUXテスト手法を用い客観的に使いやすいと言えるUIへと昇華させていく.
\par またブラウザ上の画面だけではなくアドオンを提供し解決を図ることが効果的な課題がインタビューから明らかになった. よって今後はページだけではなくブックマークアドオンも作成していくこととする.
\par 作成するアドオンについてはGoogleChromeアドオンとし,機能については
\begin{itemize}
  \item ブラウザブックマークと本サービスのブックマーク同期
  \item ブックマーク追加時に機械学習に基づいたフォルダ分けのサジェスト
  \item ブックマークへのページ内容のタグ付け
\end{itemize}
これらを中心に実装する. 実装時間が取れる場合は機能拡充及びChorome以外のブラウザのアドオン(Firefox等)も開発していく.

\subsubsection{サーバサイド}
現状フロントはモックのみで動作しているためサーバサイドとブックマークデータのやり取りが行えるように連結させていく.随時モックを実装クラスへと置き換え動作するものとしていく.
\par 機械学習を用いたページ内容の分析は現状非常に処理時間がかかる等の問題がある. アルゴリズムの改良やデータのキャッシング,DBに上がったブックマークデータを事前解析する等の工夫を施し処理時間の改善を行っていく.また検索精度は良いとお世辞にも言えない状態であるためアルゴリズムの改良を行っていく.

\subsection{今後の調査方針}
プロトタイプを改良しUI等について調査し改良していく事に加え,
サービスの料金等についてあまり深く考えられていない現状であるので,この金額で出したらサービス使う人がいるかといった事を調査していく.

\section{質問内容とインタビュー結果(後藤)}
インタビューは以下のように行った. またインタビューはできるかぎりコアユーザーにマッチするITエンジニアを対象に行った.
\begin{enumerate}
  \item 対象ユーザであるか判断する質問
  \item サービスの説明をし実際にプロトタイプを触ってもらう
  \item サービスに対する感想を求める
\end{enumerate}

\subsection{質問内容}
実際の質問内容は以下の通りである
\begin{enumerate}
  \item 対象ユーザーであるか確認する質問
  \begin{enumerate}
    \item 何をしている人か
    \item ブックマークはいくつくらいあるか
    \item ブックマークで困っていること 不満があるか
    \item 何をブックマークしているか
    \item ブックマークを整理しているか(その理由,整理する既存サービスを利用しているか)
    \item どのようにブックマークが整理できたらうれしいか
  \end{enumerate}
  \item サービスに関する質問
  \begin{enumerate}
    \item どんなサービスだと思ったか(伝わったか)
    \item このサービスを利用するか
    \item このサービスについてどう思うか(使うか)
    \begin{enumerate}
      \item 操作の回数
      \item UI
      \item 操作性
    \end{enumerate}
    \item (サービス使わないと答えた場合)この内容が改善されたら利用するか
  \end{enumerate}
\end{enumerate}
\subsection{インタビュー結果}
質問に関する回答結果は以下のようになった

\begin{figure}[htbp]
  \caption{ブックマークに関する質問}
  \begin{center}
    \includegraphics[width=14cm]{./interview-res-1.png}
  \end{center}
\end{figure}
\begin{figure}[htbp]
  \caption{サービスに関する質問}
  \begin{center}
    \includegraphics[width=14cm]{./interview-res-2.png}
  \end{center}
\end{figure}


\end{document}
