\documentclass[a4paper,10pt,fleqn]{jsarticle}
\usepackage{amsmath}
\usepackage[dvipdfm]{graphicx}


\title{プロジェクト実習テーマP  中間レポート2}
\author{
  チームF \\
  C0114312 高畑 達也\\
  C0114015 新井 幸希\\
  C0114234 後藤 尚輝\\
  C0114088 上原 安里奈\\
}
\date{\today}


\begin{document}
\maketitle

\section{システム概要(新井)}

\section{プロトタイプの説明(高畑)}

\section{顧客インタビューとその解析(上原)}

\subsection{インタビューの目的}
対象ユーザー及びコアユーザーの確認と,現在考えているサービスに価値があるのかという確認をするためにインタビューを実施した.

\subsection{インタビュー人数}
インタビューシートとプロトタイプを用いて,IT関係の人を中心に9人にインタビューを実施した.

\subsection{インタビュー解析}
インタビューを実施した人の中でブックマークの数が100個以上であったのは33\%であった.30個以上の人も合わせると78\%であり,IT関係の人はブックマークが多いと考えられる.また、ブックマークの内容に関しては,技術系ページという回答が目立った.どうブックマークを整理出来たら嬉しいかという質問では「自動分類」という回答が多く,現在のプロトタイプ,サービスに対しては「操作回数の多さ」「自動同期」「自動分類」「タグ付け」などの指摘があった.現在のサービスでは使わないと答えたのが67\%であったため、改善する必要がある.しかし「操作回数の多さ」「自動同期」「自動分類」「タグ付け」など改善が改善されれば使うとの回答が得られた事から,ザービス自体の価値はあり、ターゲットユーザーは存在すると考えられる.

\section{結論(後藤)}


\section{質問内容とインタビュー結果(後藤)}

\subsection{質問内容}
\subsection{インタビュー結果}




\end{document}
