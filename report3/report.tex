\documentclass[a4paper,10pt,fleqn]{jsarticle}
\usepackage{amsmath}
\usepackage[dvipdfmx]{graphicx}

\fboxsep=0pt
\fboxrule=1pt

\title{プロジェクト実習テーマP CloudBM  中間レポート2}
\author{
  チームF \\
  C0114312 高畑 達也\\
  C0114015 新井 幸希\\
  C0114234 後藤 尚輝\\
  C0114088 上原 安里奈\\
}
\date{\today}


\begin{document}
\maketitle

\section{システム概要(新井)}
現状ブラウザのブックマークシステムのUIは非常に貧弱であり,デフォルトの状態で大量に存在するブックマークを整理する際,多くの工数がかかってしまう.
 この乱雑に増え続けるブックマークに対する効果的なアプローチは現在存在せず,ブックマークサービスとして現存するSpeed DialやBookmark Managerのようなアドオンを使うことである程度の整理することができるが,大量のブックマーク整理に向いていない.
 上記の問題を解決するべく,我々はシンプルにブックマークを管理するCloudBMを作成した.CloudBMを使うことにより,たった数回の操作で簡単に大量のブックマークを整理することが可能となる.具体的にはフォルダ分けやタグ付けを行い,さらにリンク切れをしたブックマークの削除などを自動で行い,ユーザーはただまとめられたブックマークを検索をすれば良い.
\subsection{実装した機能}
\begin{itemize}
\item ブックマークの表示
\item ブックマークの操作
\begin{itemize}
\item DnDによるブックマークのフォルダ移動
\item DnDによるブックマークの並び替え
\end{itemize}
\item ブックマークの検索とグルーピング
\begin{itemize}
\item タイトル等からの検索
\item ブックマークのページ内容による検索
\end{itemize}
\end{itemize}

\subsection{システム構成}
以下に本システムのシステム構成を示す.
\begin{figure}[htbp]
  \begin{center}
    \fbox{\includegraphics[clip,width=7.0cm]{./architecture.png}}
    \caption{システムアーキテクチャ}
    \label{fig:system}
  \end{center}
\end{figure}
本システムはSPAとして構成する.\par
図\ref{fig:system}にしめしたように,フロントサイドではJSフレームワークとしてVuejsを用いる. またサーバーとの通信に関してはAjaxを利用する.
サーバーサイドはLinux環境内にWEBサーバーとしてApacheを,データベースとしてMySQLを用意する. ApacheではPHPを利用しフロントサイドより呼び出すAPIを実装する.\par
加えてサーバーサイドではブックマークの内容解析を行うためのAPIを用意する. このAPIはpythonとpythonのWebフレームワークであるflaskで構成されておりphpから利用される.\\
\par
使用する主なソフトウェア・言語・ライブラリのバージョンについては以下の通りとする.
\begin{itemize}
 \item フロントエンド
    \begin{itemize}
    \item HTML5
    \item CSS3
      \item Bootstrap 3.6
      \item ES5
      \item Vuejs 1.x系
     \end{itemize}
 \item サーバサイド
     \begin{itemize}
    \item PHP 7
    \item Laravel  5.1
      \item MySQL 5.6
      \item Python 3.5
      \item Flask 0.11
     \end{itemize}
\end{itemize}
\section{プロトタイプの説明(高畑)}
このサービスに対して,価値があるのかを調査するため,インタビューを行った際にインタビュー対象者へ見てもらうプロトタイプの作成を行った.実際にインタビューを行う過程でプロトタイプを見てもらい,サービスの動きを視覚的に理解してもらうことが目的である.下記の図\ref{fig:prot1}はCloudBMのプロトタイプである.

\begin{figure}[h]
  \begin{center}
    \fbox{\includegraphics[width=9.0cm]{./prot1.png}}
    \caption{プロトタイプ画面}
    \label{fig:prot1}
  \end{center}
\end{figure}

画面左側には,ブックマークフォルダのツリー構造,右側にはフォルダに格納されているブックマークが表示される.現状,動画サイトや各種SNSのページなど,様々なブックマークが表示されていて,整理出来ているとは言い難い.これらのブックマークを整理するため,画面下部の検索マークをクリックしキーワードの検索を行う.動作画面を図\ref{fig:prot2}に示す.

\begin{figure}[h]
  \begin{center}
    \fbox{\includegraphics[width=9.0cm]{./prot2.png}}
    \caption{キーワード検索画面}
    \label{fig:prot2}
  \end{center}
\end{figure}

テキストボックスにキーワードを入力することで,関連するブックマークの一覧を確認することができ,またフォルダに纏めたい場合にはボタンをクリックするだけで,関連するブックマークを一つのフォルダに簡単に整理することができる.整理した際の画面を図\ref{fig:prot3}に示す.

\begin{figure}[h]
  \begin{center}
    \fbox{\includegraphics[width=9.0cm]{./prot3.png}}
    \caption{フォルダ整理後の画面}
    \label{fig:prot3}
  \end{center}
\end{figure}

\newpage

図\ref{fig:prot3}を見て分かるように,「GitHub」というキーワードに基づくブックマークがフォルダに纏められる.整理が完了したブックマークは,アドオンを利用することにより,Webブラウザと自動同期することが出来る.このシステムを利用することにより,これまで大量に存在したブックマークを手動でフォルダ分けする必要がなくなり,より良いブラウジングが可能となる.

また,ブックマークに登録されているデータは,DnDで移動ができ,よりエクスプローラライクな操作感を実現した.また,不要になったWebページは右クリックで出現するコンテキストメニュー(図\ref{fig:prot4})より,削除をすることが出来る.

そして,付加機能として,ブックマークデータをダブルクリックすることで,即座にWebページを開くことが出来るため,ブックマークを整理するだけではなく,そのままブックマークとして使用することも可能である.

\begin{figure}[h]
  \begin{center}
    \fbox{\includegraphics[width=9.0cm]{./prot4.png}}
    \caption{コンテキストメニュー}
    \label{fig:prot4}
  \end{center}
\end{figure}

\newpage

\section{顧客インタビューとその解析(上原)}

\subsection{インタビューの目的}
対象ユーザー及びコアユーザーの確認と,サービスの適正価格がいくらかを確認し,収益予測を立てるためインタビューを実施した.

\subsection{インタビュー人数}
インタビューシートとプロトタイプを用いて,IT関係の人を中心に人にインタビューを実施した.

\subsection{インタビュー解析}
インタビューを実施した人の中でブックマークの数が100個以上であったのは33\%であった.30個以上の人も合わせると78\%であり,IT関係の人はブックマークが多いと考えられる.「サービスを利用するか」という質問では89\%が利用する,「有料サービスを利用するか」という質問では50\%が利用すると答えた.また,具体的な金額としては300円なら払うと答えた.このことからサービスとしての価値はあり,有料サービスを使うユーザは一定数いると考えられる.プロトタイプに対しては使いやすいとの回答が多かった.

\section{ビジネスを考慮したシステム提案(後藤)}
\subsection{提案するサービス}
ブラウザのブックマークをより使いやすくするサービスを提案する.
\par
現状ブラウザのブックマークのUIは機能的に貧弱であり,特に多数のブックマークに対して使いづらい. そこで本サービスではより使いやすいブックマークを実現し提供する. 具体的にはUIをエクスプローラライクなUIで置き換えエクスプローラのような利用感でブックマークを利用出来るようにする. この機能をベースとし様々な機能を拡充していく.

\subsection{想定ユーザー}
本サービスの利用対象として想定するユーザー層について述べる.
サービスの内容からブックマークに何らかの不便さを感じているユーザーが対象であるといえる.
\par そのなかからコアユーザーを決定するため,ブックマークに不便さを抱えているユーザーについて詳しく調査したところ, ブックマークが100以上存在するひとは60%近くの割合でブックマークについて強い不満を抱えておりブックマークについて何らかの解決を求めていることが明らかになった. ことからこのようなブックマークが100以上存在する人をコアユーザーと考える. このコアユーザーについてプロファイリングしたところITエンジニアが多かった. これは仕事柄WEBブラウザを頻繁に利用し調べもの等を行うことが原因と考えられる.
\par 上記より本サービスではブックマークの多いITエンジニアをコアユーザーとして考えサービスを提供していく.  

\subsection{想定市場規模}
本サービスの想定する市場について述べる.
ブックマークの利用状況について調べるためにGoogleフォームを用い54人にアンケートを行った.
その結果ブックマークの数が100を超えるユーザーは全体の14%程度であり,そのユーザーの内60%がブックマークに関して何らかの課題を抱えていることが明らかになった.
このことから本サービスの対象とするユーザーは日本国内でWEBブラウザの利用数が7000万人程度であることから590万人ほどであると予想される.

\subsection{収益方法}
有料機能と無料機能を分け月額制のサービスとして提供する.
インタビューの結果から具体的な金額として300円程度なら払うと答えた人が多かったことから300円程度で提供することを考える.
基本的な機能を無料で提供し,よりブックマークを便利に使うことができる機能を有料で提供していくことで収益を得る.
現在構想中の機能等は以下のようになる.
\begin{itemize}
  \item ブックマークをエクスプローラライクなUI経由で利用出来る(無料)
  \item ブックマークに内容タグを自動付与しタグ経由でブックマークを参照できる
  \item 内容を元にしたフォルダ分けの自動サジェスト
  \item リンク切れブックマークの抽出
\end{itemize}
このような機能を初めとしユーザーのブックマークに関して求めている機能を提供していくことで収益を得る.

\subsection{サービスの展開}
本サービスの今後の展開として,まず第一にブックマークをより便利に使うことが出来る機能の拡充を行っていく.
前項でも述べたがブックマークにタグをつけそのタグを元にしてブックマークを使うことが出来る機能を特に考えている.
現在のブックマークのデータ構造はツリー構造でありフォルダを開いてその中に存在するブックマークを開くという使い方をするが,そのような構造とは別にブックマークに内容に応じたタグを付けておき,そのタグをもとにブックマークを検索し開くというようなユースケースを実現し,新しい形でのブックマーク利用の形を提案していく.
\par
第二として,本サービスではユーザーのブックマークがサーバーと同期されるという仕様であるためサービス提供側でユーザーのブックマークを把握出来る. ブックマークはユーザーの行動や趣味と言ったパーソナリティが強く反映されるため,このデータを利用することにより高度な各ユーザーのプロファイリングが可能であると考えられる.  このデータを利用し,効果的な広告配信やページのサジェスト等を行うことが可能であり,有料サービスとして提供することに加え,このような方法でも収益を得ていくため今後機能の拡充を図っていく.

\section{質問内容とインタビュー結果}
インタビューは以下のように行った. またインタビューはできるかぎりコアユーザーにマッチするITエンジニアを対象に行った.
\begin{enumerate}
  \item 対象ユーザであるか判断する質問
  \item サービスの説明をし実際にプロトタイプを触ってもらう
  \item サービスに対する感想を求める
\end{enumerate}

\subsection{質問内容}
実際の質問内容は以下の通りである
\begin{enumerate}
  \item 対象ユーザーであるか確認する質問
  \begin{enumerate}
    \item 何をしている人か
    \item ブックマークはいくつくらいあるか
    \item ブックマークで困っていること 不満があるか
    \item 何をブックマークしているか
    \item ブックマークを整理しているか(その理由,整理する既存サービスを利用しているか)
    \item どのようにブックマークが整理できたらうれしいか
  \end{enumerate}
  \item サービスに関する質問
  \begin{enumerate}
    \item どんなサービスだと思ったか(伝わったか)
    \item このサービスを利用するか
    \item このサービスについてどう思うか(使うか)
    \begin{enumerate}
      \item 操作の回数
      \item UI
      \item 操作性
    \end{enumerate}
    \item (サービス使わないと答えた場合)この内容が改善されたら利用するか
  \end{enumerate}
\end{enumerate}
\subsection{インタビュー結果}
質問に関する回答結果を図\ref{fig:bookmarkQuestion},図\ref{fig:serviceQuestion}に示す.

\begin{figure}[htbp]
  \begin{center}
    \fbox{\includegraphics[width=14cm]{./interview-res-1.png}}
    \caption{ブックマークに関する質問}
    \label{fig:bookmarkQuestion}
  \end{center}
\end{figure}
\begin{figure}[htbp]
  \begin{center}
    \fbox{\includegraphics[width=14cm]{./interview-res-2.png}}
    \caption{サービスに関する質問}
    \label{fig:serviceQuestion}
  \end{center}
\end{figure}


\end{document}
